%-----------------------------------------------------------------------------------------------------------------
\chapter{Conclusions}
\label{ch:conc}

In conclusion, characterization of a jet engine's combustor flow field, performed in the 1990's, served as motivation for this report. The purpose was to study the effects on mixing as a result of side inlet angle variation for -20\textsuperscript{o} and 10\textsuperscript{o}.\\

A series of three CFX simulations were developed. An all triangular sweep face mesh with maximum element size of 0.001 [m] and inflated walls was used in all models. Appropriate inlet boundary conditions were assigned to both axial and side inlets for each models, based on experimental data. Mixing is simulated by assigning  axial and side entrance temperatures of 100~[\textsuperscript{o}C] and 0 [\textsuperscript{o}C], respectively. To save computational resources, symmetry boundary conditions are used. All walls were set to be smooth, no slip and adiabatic. The thermal energy heat transfer model is used since Ma $<$ 1. Due to the high Re number, the $k-\varepsilon$ turbulence model is used. A high resolution advection scheme and turbulence metrics with $t_s = 0.004$ [s] and $\delta_{r_{MAX}}=0.0001$ are used to get a solution that is $\mathcal{O}(\Delta x^2)$ accurate.\\

Simulations were performed for reference, CCW and CW angled models converging on $i= 39, 36, 44$ iterations, respectively. The reference model is validated by comparing $u,v,k$ profiles with experimental data collected at various $X^*$. Quantifying the combustor's mixing ability is done by comparing $u,v,k,T$ at various $X^*$ for each model. From this it was concluded that there is little to no effect on the quality of mixing based on the inlet angle however based on slight variations, a CCW angle or $\theta_1=$-20\textsuperscript{o} appears to provide optimal results.\\

Upon comparison with other peers from ME566 and a scholarly article, it appears as though there is no clear solution. Sources of errors attributed to the simulation results were both related to physical and numerical models. Physical model error is likely a result of the turbulence model selection. Meanwhile, numerical model error is mostly caused by the discretization of the $\mathcal{O}(\Delta x^2)$ accurate differencing scheme and grid quality.\\
