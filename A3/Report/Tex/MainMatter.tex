%\renewcommand{\chaptername}{Section}

%-----------------------------------------------------------------------------------------------------------------
%% Section 1 : Introduction
\chapter{Introduction}
\label{chap:intro}

As per \cite{assign}, Equation~\ref{eq:Gov} is presented as follows:

\begin{equation}
	\label{eq:Gov}	
	\frac{\partial(uT)}{\partial x} + \frac{\partial(vT)}{\partial y}= \frac{\partial}{\partial x} \left( \alpha \frac{\partial T}{\partial x} \right) + \frac{\partial}{\partial y} \left( \alpha \frac{\partial T}{\partial y} \right)
\end{equation}

Yielding the following solution ($a \cdot T = Su$) using CDS as per \cite{assign} and previously completed assignment.

%equation * creates no number
\begin{equation*}
	\begin{bmatrix}
		78.6	&	-0.4	&	-25.0 	&	0		\\
		-1.6	&	76.6 	&	0  		&	-25.0	\\
		-25.0	&	0   	&	78.6	&	-0.4	\\
		0  		&	-25.0	&	-1.6	&	76.6
	\end{bmatrix}
	\centerdot 
	\begin{pmatrix} T_1	\\	T_2	\\	T_3	\\ T_4	\end{pmatrix}
	=
	\begin{pmatrix} 5064	\\	5000	\\	2564	\\ 2500 \end{pmatrix}	
\end{equation*}

The coefficient matrix $a$ and source term vector $Su$ are used next to solve the system of equations iteratively using three different indirect methods.

%-----------------------------------------------------------------------------------------------------------------
%% Section 2 : Jacobi
\chapter{Jacobi Method}
\label{chap:jacobi}

Assuming the initial guesses $k=0$ for values of T are zero (i.e. $T_1^0=T_2^0=T_3^0=T_4^0=0$), the Jacobi Method (JM) algorithm in Equation~\ref{eq:jacobi} \cite{cfdbook} is implemented numerically (see script \cite{python} in Appendix).

\begin{equation}
	\label{eq:jacobi}	
	T_i^k = \frac{1}{a_{ii}} \left( \sum_{j=1, j \neq i}^{n} - a_{ij} T_j^{k-1}  + Su_i \right)
\end{equation}

The results for $k=3$ iterations are shown below in Table~\ref{tab:jac}.

\begin{table}[H]
  \centering
  \caption{Results for $k=3$ iterations using JM.}
    \begin{tabular}{cccc}
    & \textbf{$k=1$} & \textbf{$k=2$} & \textbf{$k=3$} \\
    \midrule
    \textbf{$T_1$} & 64.4275 & 75.1353 & 81.7670 \\
    \textbf{$T_2$} & 65.2742 & 77.2717 & 84.6706 \\
    \textbf{$T_3$} & 32.6209 & 53.2792 & 56.7968 \\
    \textbf{$T_4$} & 32.6371 & 54.6220 & 58.9692 \\
    \end{tabular}
  \label{tab:jac}
\end{table}

After three iterations, the JM yields results $\approx 2-3 \deg$C from the exact solution provided in \cite{assign}. This error will be computed in Section~\ref{chap:error}.

%-----------------------------------------------------------------------------------------------------------------
%% Section 3 : GS
\chapter{Gauss-Seidel Method}
\label{chap:gauss}

The Gauss-Seidel Method (GSM) is impleted with the algorithm in Equation~\ref{eq:gs} \cite{cfdbook}. This is implement in the script found in the Appendix.

\begin{equation}
	\label{eq:gs}	
	T_i^k = \frac{1}{a_{ii}} \left( \sum_{j=1}^{i-1} - a_{ij} T_j^k  + \sum_{j=i+1}^{n} - a_{ij} T_j^{k-1} + Su_i \right)
\end{equation}

Results for all $k=3$ iterations are summarized in Table~\ref{tab:gs} below.

\begin{table}[H]
  \centering
  \caption{Results for $k=3$ iterations using GS.}
    \begin{tabular}{cccc}
          & $k=1$ & $k=2$ & $k=3$ \\
    \midrule
    $T_1$ & 1.0000 & 1.0000 & 1.0000 \\
    $T_2$ & 1.0000 & 1.0000 & 1.0000 \\
    $T_3$ & 1.0000 & 1.0000 & 1.0000 \\
    $T_4$ & 1.0000 & 1.0000 & 1.0000 \\
    \end{tabular}
  \label{tab:gs}
\end{table}

Relative to the method previously used, these results are.

%-----------------------------------------------------------------------------------------------------------------
%% Section 4 : Relax
\chapter{Relaxation Method}
\label{chap:relax}


The final iterative method explored in this assignement is the Relaxtion Method (RM). The algorithm in Equation~\ref{eq:rlx} \cite{cfdbook} is seen in the script from the Appendix.

\begin{equation}
	\label{eq:rlx}	
	% T_i^k = T_i^{k-1} + \frac{\omega}{a_{ii}} \left( \sum_{j=1}^{n} - a_{ij} T_j^{k-1} + Su_i \right)
	T_i^k = \omega T_{i_{GS}}^k +(1-\omega)T_i^{k-1}
\end{equation}

Again, results for each iteration are summarized below (see Table~\ref{tab:rm}).

\begin{table}[H]
  \centering
  \caption{Results for $k=3$ iterations using RM.}
    \begin{tabular}{cccc}
          & $k=1$ & $k=2$ & $k=3$ \\
    \midrule
    $T_1$ & 1.0000 & 1.0000 & 1.0000 \\
    $T_2$ & 1.0000 & 1.0000 & 1.0000 \\
    $T_3$ & 1.0000 & 1.0000 & 1.0000 \\
    $T_4$ & 1.0000 & 1.0000 & 1.0000 \\
    \end{tabular}
  \label{tab:rm}
\end{table}



%-----------------------------------------------------------------------------------------------------------------
%% Section 5 : Error
\chapter{Error}
\label{chap:error}

The error term for each method is computed with Equation~\ref{eq:err} as per \cite{assign}.

\begin{equation}
	\label{eq:err}	
	e = |T_1 - T_{1_{ex}}|+|T_2 - T_{2_{ex}}|+|T_3 - T_{3_{ex}}|+|T_4 - T_{4_{ex}}|
\end{equation}

Where $T_1 = 83.8295, T_2 = 87.3921, T_3 = 59.6018, T_4 = 62.4042$. The error $e$ was computed for the JM, GSM and RM. The results are summarized below in Table~\ref{tab:err}.

\begin{table}[H]
  \centering
  \caption{Errors for each iterative method.}
    \begin{tabular}{c|c}
    $e_{JM}$ 	& 11.0240 \\
    $e_{GSM}$ 	& 0.751X \\
    $e_{RM}$ 	& 0.130X \\
    \end{tabular}
  \label{tab:err}
\end{table}

From these results, it may be concluded that the relaxation iterative method is the most efficient. The over relaxation  (i.e. $\alpha>1$) results in an accelerated GSM hence more accurate results. 


